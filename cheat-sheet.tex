\documentclass[accentcolor=tud8b,colorbacktitle,12pt]{tudexercise}


\usepackage[T1]{fontenc}
\usepackage[utf8]{inputenc}
\usepackage[ngerman]{babel}
\usepackage{hyperref}
\usepackage{ifthen}
\usepackage{listings}
\usepackage{graphicx}
\usepackage{multicol}
\usepackage{multirow}
\usepackage{tabto}

\definecolor{darkblue}{rgb}{0,0,.5}
\hypersetup{colorlinks=true, breaklinks=true, linkcolor=darkblue, menucolor=darkblue, urlcolor=darkblue}

\newcommand{\lstinlinenoit}[1]{\upshape{\lstinline|#1|}\itshape}
\lstset{language=sh,basicstyle=\ttfamily\small, keywordstyle=\color{blue!80!black}, identifierstyle=, commentstyle=\color{green!50!black}, stringstyle=\ttfamily,
 tabsize=4, breaklines=true, numbers=left, numberstyle=\small, frame=single, backgroundcolor=\color{blue!3}}
 
\author{Fachschaft Informatik | Fachgebiet Data Management, TU Darmstadt}


\begin{document}
\title{Versionsverwaltung mit Git\\Befehlsreferenz}
\subtitle{Fachschaft Informatik | Fachgebiet Data Management, TU Darmstadt}
\subsubtitle{Stand: \today}
\maketitle

\hskip 1 pt\\
\textbf{Alle Befehle müssen in einem Git-Repository ausgeführt werden!}\\
\textit{\lstinline|<name>| stellt einen Platzhalter dar. z.B. \lstinline|git add <datei>| $\rightarrow$ \lstinline|git add datei.txt|}
\section*{Einrichtung}
\begin{itemize}
	\item Setze \textit{Author}-Informationen:\\
	\lstinline|git config --global user.email "you@example.com"|\\
	\lstinline|git config --global user.name "Your Name"|

	\item Setze einen alternativen anfängerfreundlichen Editor:
	\NumTabs{7}
	\begin{itemize}
		\item Linux: \tab\lstinline|git config --global core.editor nano|
		\item macOS: \tab\lstinline|git config --global core.editor textedit|
		\item Windows: \tab\lstinline|git config --global core.editor notepad|
	\end{itemize}
\end{itemize}

\section*{Begriffe}
\NumTabs{7}
\begin{itemize}
	\item \textbf{Repository}\tab ein Git-Verzeichnis
	\item \textbf{Commit}\tab ein Paket abgeschlossener Änderungen in der Historie
	\item \textbf{ref}\tab z. B. ein \textit{Commit}
\end{itemize}

\section*{Allgemein}
\begin{itemize}
	\item \lstinline|git init|: Anlegen eines Git-Repositories. \textbf{Nicht} erforderlich, wenn man ein Repository klont (siehe Zusammenarbeit).
	\item \lstinline|git status|: Zeigt an, welche Änderungen im Repository für den Commit vorgemerkt sind.
	\item \lstinline|git add <datei>|: Fügt Änderungen zu einem Commit hinzu.
	\item \lstinline|git rm <datei>|: Löscht eine Datei sowohl aus dem Verzeichnis als auch vom Repository.
	\item \lstinline|git commit|: Schreibt Änderungen in die Historie.
	\item \lstinline|git commit --amend|: Schreibt den letzten Commit um (Beschreibung, Änderungen, etc.).
	\item \lstinline|git log|: Zeigt die Historie/Commits an.
	\item \lstinline|git show <ref>|: Zeigt Informationen zu einer bestimmten ref an (inkl. Änderungen).
	\item \lstinline|git checkout <ref>|: Setzt das Git-Repository auf eine bestimmte ref.
	\item \lstinline|git diff <ref>..<ref>|: Zeigt die Änderungen (pro Datei) von einer ref zu einer anderen ref an.
	\item \lstinline|git revert <commit>|: Macht einen Commit rückgängig (erstellt einen neuen Commit, welcher den alten invertiert).
\end{itemize}

\section*{Zusammenarbeit}
\begin{itemize}
	\label{CollabClone}
	\item \lstinline|git clone <url>|: Klont das Git-Repository von einem Server
	\item \lstinline|git push|: Lädt Commits auf den Server hoch
	\item \lstinline|git fetch --all|: Lädt Änderungen aller Branches vom Server herunter
	\item \lstinline|git pull|: Lädt Änderungen vom Server herunter und merged ggf. automatisch oder produziert Merge-Konflikte
	\item \lstinline|git branch|: Listet alle Branches des Git-Repos auf
	\item \lstinline|git branch <branch>|: Erstellt Branch branch; Parameter -d löscht ihn (-D auch wenn noch nicht gemergt)
	\item \lstinline|git checkout <branch>|: Wechselt zum Branch branch; Parameter -b erstellt den Branch dabei
	\item \lstinline|git merge <branch>|: Merged Branch branch in aktuellen Branch
\end{itemize}

\section*{.gitignore}
Wenn man manche Dateien nicht versionieren möchte, kann man diese auf eine Ignorierliste setzen. Dafür wir eine \lstinline|.gitignore|-Datei verwendet.\\
\textbf{Achtung:} Diese Datei hat keinen Dateinamen, nur eine Dateiendung. Unter Windows ist die Erstellung solch einer Datei im Windows-Explorer nicht möglich. Dafür muss die Datei direkt aus Notepad gespeichert werden, wähle dabei \textbf{keine} Dateiendung aus der Liste aus.\\
Die Datei enthält die Namen von zu ignorierenden Dateien, dabei können auch Verzeichnisnamen oder "`Wildcards"' verwendet werden.\\
\\
Beispielhafter Inhalt:
\lstinputlisting{example.gitignore}
\end{document}
