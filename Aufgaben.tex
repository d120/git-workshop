\documentclass[accentcolor=tud8b,colorbacktitle,12pt]{tudexercise}
\usepackage[T1]{fontenc}
\usepackage[utf8]{inputenc}
\usepackage[ngerman]{babel}
\usepackage{hyperref}
\usepackage{ifthen}
\usepackage{listings}
\usepackage{graphicx}
\usepackage{multicol}
\usepackage{multirow}

\definecolor{darkblue}{rgb}{0,0,.5}
\hypersetup{colorlinks=true, breaklinks=true, linkcolor=darkblue, menucolor=darkblue, urlcolor=darkblue}

\newcommand{\lstinlinenoit}[1]{\upshape{\lstinline|#1|}\itshape}
\lstset{language=Java, basicstyle=\ttfamily\small, keywordstyle=\color{blue!80!black}, identifierstyle=, commentstyle=\color{green!50!black}, stringstyle=\ttfamily,
 tabsize=4, breaklines=true, numbers=left, numberstyle=\small, frame=single, backgroundcolor=\color{blue!3}}

% generate solutions
\newboolean{sln}\setboolean{sln}{True}
\newcommand{\sln}[1]{\ifthenelse{\boolean{sln}}{\subsubsection*{Antwort}{\itshape #1}}{}}
\newcommand{\task}[1]{\input{task/#1}\IfFileExists{./task/#1}{\sln{\input{sln/#1}}}{\ClassError{Vorkurs-TeX}{No solution specified for task #1}{Add solution file #1.tex or #1.java}}}


\begin{document}
\title{Versionsverwaltung mit Git}
\subtitle{Fachschaft Informatik, TU Darmstadt}
\subsubtitle{Stand: \today}
\author{Fachschaft Informatik, TU Darmstadt}
\maketitle

\section{Einzelarbeit}
\setcounter{subsection}{-1}
\subsection{Installation \& Einrichtung}
\begin{enumerate}
	\item Git-Installer für Windows wie für macOS liegen lassen sich von \href{https://git-scm.com/}{https://git-scm.com/} herunterladen. Linux-Distributionen haben Git in der Regel in den Paketquellen.
	\item Folge den Anweisungen der Installation. Bei Fragen, frag uns ;-).
	\item Starte eine Git-Shell:
	\begin{itemize}
		\item Windows: Suche Git über das Startmenü ODER wähle in einem beliebigem Verzeichnis: \lstinline|Open Git Bash here|
		\item Linux, macOS: Öffne ein(e) Terminal/Shell/Konsole/...
	\end{itemize}

	\item Setze entsprechend der Git-Befehlsreferenz \textbf{Editor} und \textbf{Autor}.
\end{enumerate}

\subsection{Erste Schritte}
Schreibe eine simple Funktion \lstinline{int add(int a, int b)}, welche zwei Zahlen aufaddiert.
Erstelle nun einen Commit mit deinen Änderungen.

\task{firstCalc}

\task{otherSub}

\section{Zeitreise}
Es ist Weihnachten. Draußen schneit es. Bei Gebäck und Glühwein denkst du an Java und deinen Erfolg aus Aufgabe 1. Du möchtest nun noch einmal sehen wie sich dein Taschenrechner in der Zeit entwickelt hat.

\task{poi}

\task{backtofuture}

\section{Gruppenarbeit}
Finde eine*n Partner*in.
Legt nun im \href{https://scm.informatik.tu-darmstadt.de}{ISP-SCM} ein gemeinsames Projekt mit Git-Repository an.
\task{Parallelism}

\task{smartarse}

\end{document}
